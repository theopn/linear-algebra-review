%! TEX root = ../ma351.tex


% Following sections from my another note, probably belongs to this section
\section{Checking Invertibility of a Matrix}

Following statements are equivalent for an $n \times n$ matrix $A$.

\begin{itemize}
  \item $A$ is invertible (or nonsingular), i.e., there exists a matrix $B$ such that $AB = I_n = BA$
  \item $\mathrm{Rank} A = n$
  \item $A$ is row equivalent to $I_n$
  \item The dimensions of both row space and column space of $A$ are $n$
  \item The nullspace (kernel) of $A$ is $\{0\}$
  \item $\forall B \in \mathbb{R}^n$, $AX = B$ has exactly one solution
  \item $\mathrm{Det} A \neq 0$
\end{itemize}

Let us invert $A$ where

\begin{align*}
  A =
  \begin{pmatrix}
    2 & 0 & -1\\
    5 & 1 &  0 \\
    0 & 1 &  3
  \end{pmatrix}
\end{align*}

To verify the Invertibility of $A$, we can check its determinant.

\begin{align*}
  \begin{pmatrix}
    2 & 0 & -1\\
    5 & 1 &  0 \\
    0 & 1 &  3
  \end{pmatrix}
  \begin{pmatrix}
    2 & 0 &           -1\\
    0 & 1 & \frac{5}{2} \\
    0 & 1 &           3
  \end{pmatrix}
  \explain{$R_2 \rightarrow R_2 - \frac{5}{2} R_1$}
  \begin{pmatrix}
    2 & 0 &           -1\\
    0 & 1 & \frac{5}{2} \\
    0 & 0 & \frac{1}{2}
  \end{pmatrix}
  \explain{$R_3 \rightarrow R_3 - R_2$}\\
  \therefore \det A = 2 \cdot 1 \cdot \frac{1}{2} = 1
\end{align*}

Thus, $A$ is invertible.

\section{Using Row Operations to Invert a Matrix}

We can reduce $[A | I]$ until it becomes $[I | A^{-1}]$ to find the inverse.
By the way, did you know that $I = [\delta_{ij}]$ where $\delta_{ij}$ is the Kronecker delta defined by

\begin{align*}
  \delta_{ij} =
  \begin{cases}
    0 & \quad i \neq j\\
    1 & \quad i = j
  \end{cases}
\end{align*}

Anyway, reducing $[A | I]$,

\begin{align*}
  \left(
    \begin{array}{rrr|rrr}  % r for right, c for center
      2 & 0 & -1 & 1 & 0 & 0\\
      5 & 1 &  0 & 0 & 1 & 0\\
      0 & 1 &  3 & 0 & 0 & 1
    \end{array}
  \right)\\
  \left(
    \begin{array}{rrr|rrr}
      1 & 0 & - \frac{1}{2} & \frac{1}{2} & 0 & 0\\
      5 & 1 &             0 &           0 & 1 & 0\\
      0 & 1 &             3 &           0 & 0 & 1
    \end{array}
  \right)
  \explain{$R_1 \rightarrow \frac{R_1}{2}$}\\
  \left(
    \begin{array}{rrr|rrr}
      1 & 0 & - \frac{1}{2} &  \frac{1}{2} & 0 & 0\\
      0 & 1 &   \frac{5}{2} & -\frac{5}{2} & 1 & 0\\
      0 & 1 &             3 &            0 & 0 & 1
    \end{array}
  \right)
  \explain{$R_2 \rightarrow R_2 - 5R_1$}\\
  \left(
    \begin{array}{rrr|rrr}
      1 & 0 & - \frac{1}{2} &  \frac{1}{2} &  0 & 0\\
      0 & 1 &   \frac{5}{2} & -\frac{5}{2} &  1 & 0\\
      0 & 0 &   \frac{1}{2} &  \frac{5}{2} & -1 & 1
    \end{array}
  \right)
  \explain{$R_3 \rightarrow R_3 - R_2$}\\
  \left(
    \begin{array}{rrr|rrr}
      1 & 0 &             0 &            3 & -1 &  1\\
      0 & 1 &             0 &          -15 &  6 & -5\\
      0 & 0 &   \frac{1}{2} &  \frac{5}{2} & -1 &  1
    \end{array}
  \right)
  \explain{{$R_1 \rightarrow R_1 + R_3$}\\{$R_2 \rightarrow R_2 - 5R_3$}}\\
  \left(
    \begin{array}{rrr|rrr}
      1 & 0 & 0 &   3 & -1 &  1\\
      0 & 1 & 0 & -15 &  6 & -5\\
      0 & 0 & 1 &   5 & -2 &  2
    \end{array}
  \right)
  \explain{$R_3 \rightarrow 2R_3$}
\end{align*}

Thus,

\begin{align*}
  A^{-1} =
  \begin{pmatrix}
    3 & -1 & 1\\
    -15 & 6 & -5\\
    5 & -2 & 2
  \end{pmatrix}
\end{align*}

And indeed, $AA^{-1} = I$.

